\documentclass[12pt]{article}

\usepackage{lmodern}
\usepackage[T1]{fontenc}
\usepackage[spanish,activeacute]{babel}
\usepackage[utf8]{inputenc}
\usepackage{mathtools}
\usepackage{enumerate}
\usepackage{amsthm}
\usepackage{amssymb}
\usepackage{float}
\usepackage{subfig}
\usepackage{anysize}
\usepackage{wrapfig}

\marginsize{2cm}{2cm}{2cm}{2cm}

\title{Temas Ingeniería, Empresa y Sociedad}
\author{ }

\newtheoremstyle{definition_wo_parentheses}
  {\topsep}% measure of space to leave above the theorem. E.g.: 3pt
  {\topsep}% measure of space to leave below the theorem. E.g.: 3pt
  {}% name of font to use in the body of the theorem
  {0pt}% measure of space to indent
  {\bfseries}% name of head font
  {.}% punctuation between head and body
  { }% space after theorem head; " " = normal interword space
  {\thmname{#1}\thmnumber{ #2.}\thmnote{ #3}}
  
\theoremstyle{definition_wo_parentheses}	
\newtheorem{definicion}{Definición}[section]

\begin{document}
\maketitle

\section{Tema 1: La empresa y la dirección de empresas}
\textbf{Organización}: Unidad coordinada formada por un mínimo de dos personas que trabajan para alcanzar un objetivo o conjunto de objetivos comunes.\\
\textbf{Empresa}: Organizaciones que proveen bienes o servicios cuya finalidad es la obtención de beneficios (diferencia entre ingresos y gastos).\\

\subsection{Los subsistemas funcionales de la empresa}
La empresa como sistema puede ser descompuesta en subsistemas que poseen las características del sistema general. Desde una perspectiva tradicional proponemos la división de la empresa en subsistemas funcionales de: financiación, marketing, producción, investigación, desarrollo, personal, etc.\\
\textbf{Crítica}: suponer que tales subsistemas constituyen unidades aisladas, cuando en realidad están en continua interacción.\\
\textbf{Solución}: Subsistema management (administrativo).\\

\textsc{El subsistema management} tiene diferentes funciones:
\begin{enumerate}
\item \textbf{Función general}: Integrar las distintas partes y elementos de la empresa entre sí e integrar la empresa con su entorno.
\item \textbf{Funciones específicas}: 
\begin{enumerate}
\item Planificación: Implica la definición de objetivos, conductas y acciones concretas para alcanzarlos tanto a nivel global como a nivel de los diferentes subsistemas con la finalidad de proyectar la empresa en el futuro.
\item Organización: Tiene como objetivo establecer un orden interno coherente que permita a la empresa funcionar con unidad dentro y frente a su entorno. Implica la estructuración de las relaciones interpersonales y la integración y coordinación del esfuerzo de todos los miembros a pesar de intereses divergentes.
\item Dirección: Liderar, motivar y gestionar grupos.
\item Control: Debido a la naturaleza abierta del sistema empresa, es un complemento necesario para la planificación. (Controlar que todo vaya según lo previsto).
\end{enumerate}
\end{enumerate}

\textbf{Funciones gerenciales secuenciales}.\\
La \underline{planificación} (definir metas, estableces estrategias y desarrollar subplanes para coordinar las actividades), la \underline{organización} (determinar qué debe hacerse, cómo se hará y quién deberá hacerlo), la \underline{dirección} (dirigir y motivar a los participantes y resolver conflictos) y el \underline{control} (vigilar las actividades para asegurarse de que se cumplan conforme a lo planeado), en este orden, conducen a alcanzar el propósito establecido de la organización.\\

Sin embargo el proceso administrativo tiene una \textbf{naturaleza interactiva} y todas las tareas se relacionan entre sí: en \underline{planificación}, los gerentes usan la lógica y los métodos para analizar metas y acciones; en \underline{organización}, los gerentes ordenan y asignan el trabajo, la autoridad y los recursos para alcanzar las metas de la organización; en \underline{dirección}, los gerentes dirigen, influyen y motivan a los empleados para que realicen las tareas esenciales y en \underline{control}, los gerentes se aseguran de que la organización se dirige hacia los objetivos de la organización.

\section{Tema 2: El empresario}
\textbf{Empresario}: Persona o grupo de personas (órgano colegiado) que da vida a la empresa: coordina, dirige y controla el proceso productivo.

\subsection{La dirección: Funciones y niveles}
\begin{enumerate}
\item \textbf{Diferenciación vertical}: Compuesta por una alta dirección, directivos medios y supervisores de primera línea. 
\item \textbf{Diferenciación horizontal}: Compuesta por directivos generalistas arriba y directivos generalistas y directivos funcionales por debajo de los primeros directivos generalistas.
\end{enumerate}

	Como ejemplo, si una empresa tiene varias secciones para diferentes productos, pero sólo tiene un departamento, por ejemplo, de publicidad, legal o económico común a todas las secciones, será diferenciación vertical. Por el contrario, si cada sección tiene un departamento propio para estos asuntos, será diferenciación horizontal.

\begin{figure}[H]
 \centering
  \subfloat[Diferenciación Horizontal]{
   \label{f:difH}
    \includegraphics[width=0.5\textwidth]{difHorizontal}}
  \subfloat[Diferenciación Vertical]{
   \label{f:difV}
    \includegraphics[width=0.5\textwidth]{difVertical}}
 \caption{Ejemplo diferenciación}
 \label{f:dif}
\end{figure}


\section{Tema 4: Organización}

\subsection{Introducción a la función de organización}

\begin{definicion}[Organización] 
	Diseño del armazón material y humano que actuará de soporte para la ejecución de los planes establecidos.
\end{definicion}

\begin{itemize}
\item El punto de partida es el conocimiento de la misión, así como de los objetivos (estratégicos y operativos).

\item Utilización de diferentes herramientas (parámetros de diseños para construir la estructura. Implica tomar decisiones internas sobre: 

	\begin{itemize}
		\item Diseño de puestos de trabajo (especialización, formalización y preparación).
		\item Agrupación de puestos en unidades (departamentalización)
		\item Tamaño de las unidades.
		\item Establecimiento de vínculos laterales para coordinar departamentos.
		\item Descentralización de la toma de decisiones.
	\end{itemize}
	
\item Consideración de la incidencia de los factores de contingencia en los parámetros del diseño. Son factores de contingencia la edad y el tamaño de la empresa, el sistema técnico y el entorno
\end{itemize}

\subsection{Diseño organizativo}

\subsubsection{Mecanismos de coordinación}

\paragraph{Adaptación mutua.} Consigue la coordinación del trabajo mediante la simple comunicación informal. El control corre a cargo de quienes lo realizan. Asumible cuando los grupos son pequeños.

\paragraph{Supervisión directa.} Una persona es responsable del trabajo de los demás. Más frecuente conforme los grupos crecen.

\paragraph{Normalización.} Coordinación incorporada en el programa de trabajo. Menor necesidad de una comunicación continua. Son mecanismos adquiridos, facilita el crecimiento de los grupos. 

\subsubsection{Partes de la organización}

Entre las partes de la organización se distinguen las unidades de líneas; trabajan en la empresa, en el servicio que presta la empresa; y las unidades de \textit{staff}, que no trabajan para el objetivo de la empresa, sino que ofrecen servicios necesarios para ésta, de apoyo a la organización, no participan en la producción.

Las unidades de línea son:
\paragraph{Núcleo de operaciones.} Obreros. Miembros de la organización que realizan el trabajo básico directamente relacionado con la producción de bienes y servicios. Sus funciones son asegurar la entrada del material para la producción, su transformación, distribución de la mercancía. Constituyen el centro de toda organización. Es donde la normalización se aplica con mayor profundidad.

\paragraph{Ápice estratégico.} Es el órgano que se ocupa de que la organización cumpla, efectivamente, con su misión y que satisfaga convenientemente los intereses de los grupos y personas involucradas en la misma (socios, Estado, sindicatos, etc.)\\
Sus funciones son supervisar la organización y velar por que funcione debidamente como una unidad integrada, mantener relaciones con el entorno y desarrollar las estrategias de la organización. Desempeña un papel vital en la formulación de la estrategia. Su trabajo se caracteriza por un mínimo de repetición y normalización. La adaptación mutua es el mecanismo habitual.

\paragraph{Línea media.} Abarca desde los mandos situados bajo el ápice estratégico hasta los supervisores de primera línea (los que ejercen la supervisión directa sobre los operarios). Sus funciones son enlazar el ápice estratégico con el núcleo de operaciones, transmitiendo y ejecutando decisiones e información.

Las unidades de \textit{staff} se dividen en:

\paragraph{Tecnoestructura o analistas.} Formada por los analistas (y su personal administrativo) que sirven a la organización operando sobre el trabajo de los demás miembros de la misma. No intervienen directamente en el flujo de operaciones, aunque lo diseñan, planifican, cambian y preparan a las personas que lo realizan.

\paragraph{\textit{Staff} de apoyo.} Unidades especializadas cuya función consiste en proporcionar asistencia a la organización fuera del flujo de trabajo de operaciones corrientes. Estos servicios se pueden subcontratar o ser gestionados por la organización, controlándolos directamente. Estas unidades pueden funcionar como miniorganizaciones, con su propio núcleo de operaciones, línea media y ápice estratégico. Son ejemplos de \textit{staff} de apoyo servicios como el de cafetería o limpieza de una empresa que se dedique a otro sector.

	El componente administrativo de la organización está compuesto por el ápice estratégico, la línea media y la tecnoestructura. 
	
	
\subsection{Dimensiones del diseño organizativo}

\subsubsection{Diseño de puestos}

\begin{definicion}[Diseño de puestos]
Es el proceso mediante el cual se enseñan las habilidades y conocimientos relacionados con el puesto. 
\end{definicion}

	Existen dos tipos de puestos de trabajo:
	
\begin{enumerate}
\item No cualificado. Al realizarse un trabajo sumamente racionalizado, supone una extensa especialización tanto horizontal como vertical, siendo a menudo controlado y coordinado por la formalización directa.

\item Profesional. Corresponde a un trabajo que no puede especializarse fácilmente en la dimensión vertical ni ser formalizado por la tecnoestructura de la organización, teniendo una especialización horizontal y consiguiéndose a menudo una coordinación mediante la normalización de habilidades en exhaustivos programas de preparación, que suelen impartirse fuera de la organización.
\end{enumerate}

\paragraph{Formalización del comportamiento} La formalización del comportamiento representa la forma en la que la organización limita la libertad de acción. Mediante este parámetro se normalizan los procesos de trabajo. La formalización se refiere a la existencia de descripciones explícitas o implícitas relativas a reglas, procedimientos y procesos de toma de decisiones, de comunicación de instrucciones y de transmisión de información que indican en todo momento lo que ha de hacer el trabajador.\\
La formalización la realizan los analistas del \textit{staff}, lo que implica para los trabajadores una pérdida de control de sus actividades, dando lugar a una especialización vertical del puesto. La formalización del trabajo es mayor en los niveles operativos, ya que en estos se realizan las actividades más sencillas y repetitivas (especialización horizontal). Las organizaciones que se basan ante todo en la formalización del comportamiento para conseguir una coordinación suelen denominarse burocracias.

\begin{enumerate}
\item Formas burocráticas. Basada en la formalización del comportamiento. Son organizaciones con circunstancias estables. Basada en la normalización
\item Formas orgánicas. Relaciones de trabajo abiertas e informales. Organizaciones que necesitan innovaciones. Basada en la adaptación mutua.
\end{enumerate}

Definiremos la estructura orgánica como la ausencia de normalización. 

\begin{definicion}[Especialización del trabajo] 
	La especialización es la destreza que consigue un individuo para la realización de unas tareas.
\end{definicion}

Los trabajos sumamente repetitivos se tienden a dividir cada vez más en tareas más sencillas para que el trabajador las aprenda antes y las realice de forma más rápida. La especialización se realiza en dos aspectos:

\begin{enumerate}[a]
\item Especialización horizontal. Realización de pocas tareas o con poco contenido.
\item Especialización vertical. Existe poco control sobre el trabajo. La especialización vertical del puesto separa la realización del trabajo y la administración del mismo
\end{enumerate}

Lo contrario de la especialización del puesto es la ampliación.

\subsubsection{Diseño de la superestructura}

El siguiente paso es agrupar los puestos de trabajo que hemos diseñado en unidades, y diseñar el tamaño de cada unidad. Esto da los parámetros de diseño \underline{agrupación de unidades} y \underline{tamaño de la unidad}.


\begin{description}
\item [Agrupación de unidades] Las bases de agrupación son: Por conocimientos y habilidades, según el proceso de trabajo y la función, según el \textit{output}, por clientes o por zonas geográficas. Siguiendo la clasificación de Mintzberg, las dos primeras son agrupaciones funcionales y las restantes agrupaciones según el mercado.

\item [Tamaño de la unidad] El tamaño de la unidad es determinado por el concepto de ámbito o tramo de control
\end{description}


\begin{definicion}[Ámbito o tramo de control] Número de subordinados que un gerente puede supervisar de manera eficaz. Este concepto determina el número de niveles y gerentes de una organización.
\end{definicion}

\begin{description}
\item [Estructuras altas] pequeñas unidades y tramos de control estrechos.
\item [Estructuras planas] grandes unidades y tramos de control amplios.
\end{description}


\subsubsection{Diseño del sistema decisor}

\begin{definicion}[Centralización] 

	Describe dónde está la autoridad para la toma de decisiones.
	
\end{definicion}

	Pese a que la centralización coordina la toma de decisiones, la descentralización tiene como ventajas el reparto de la responsabilidad, la rapidez en la toma y constituye un estímulo a la motivación.
	
\subsection{Los factores de contigencia}

\paragraph{Edad y tamaño} Cuanto más antigua y mayor, habrá una mayor formalización y la estructura será más compleja.

\paragraph{Sistema técnico} Cuanto más regulador sea el sistema técnico, habrá mayor formalización. Cuanto más sofisticado, habrá una mayor descentralización selectiva.

\paragraph{El entorno} Cuanto más dinámico, más orgánica resulta la estructura. Cuanto más complejo, más descentralizada. Disparidades en el entorno estimulan la descentralización.

\subsection{Configuraciones estructurales}

\begin{definicion}[Eficacia]
Congruencia interna de las variables de diseño. Congruencia de dichas variables con los factores de contingencia.
\end{definicion}

Las configuraciones estructurales se diferencian por el mecanismo de coordinación predominante y por el grado de centralización:

\begin{enumerate}[I]
\item Estructuras orgánicas $\rightarrow$ Ausencia de normalización.
\begin{enumerate}[a]
\item Estructuras orgánicas centralizadas: \underline{Estructuras simples}
\item Estructuras orgánicas descentralizadas: \underline{Adhocracias}
\end{enumerate}
\item Estructuras burocráticas $\rightarrow$ Normalización.
\begin{enumerate}[a]
\item Estructuras burocráticas centralizadas: \underline{B. maquinales}
\item Estructuras burocráticas descentralizadas: \underline{B. profesionales}
\end{enumerate}
\end{enumerate}


\begin{description}
\item[Estructura simple] Se caracteriza por la falta de elaboración. Dispone de una tecnoestructura mínima o nula, reducido \textit{staff} de apoyo, división poco estricta del trabajo, diferenciación mínima entre unidades y una pequeña jerarquía directiva. Presenta poco comportamiento formalizado. Es principalmente orgánica. Coordinación mediante supervisión directa. La estructura consiste a menudo en un ápice estratégico de una sola persona y un núcleo de operaciones orgánico.
\item[Burocracia maquinal] Se caracteriza por tareas de operaciones altamente especializadas y rutinarias, procedimientos sumamente formalizados en el núcleo de operaciones, una proliferación de reglas, normas y comunicación formal a través de toda organización, unidades de gran tamaño en el núcleo de operaciones, tareas agrupadas a base de su función, poder de decisión relativamente centralizado, elaborada estructura administrativa con una clara distinción entre línea y \textit{staff}. La burocracia maquinal genera sus normas y recurre a la autoridad de naturaleza jerárquica.

\item[Burocracia profesional] Cuenta para su coordinación con la normalización de las habilidades y su correspondiente parámetro de diseño, la preparación. Contrata a especialistas debidamente preparados para su núcleo de operaciones, confiriéndoles a continuación un control considerable sobre su propio trabajo. Estructura sumamente descentralizada. El núcleo de operaciones constituye la parte central. Tecnoestrucura y línea media no muy elaboradas. Las normes surgen, por regla general, fuera de su propia estructura. Hace hincapié en la autoridad de naturaleza profesional.

\item[Adhocracia] Nionguna de las estructuras anteriores son capaces de realizar una innovación sofisticada. La estructura simple innova pero de forma sencilla. Las burocracias son estructuras de rendimiento y no de solución de problemas. Es orgánica, tiene escasa formalización. Elevada especialización horizontal del puesto basada en la preparación. Tendencia a agrupar a los especialistas en unidades funcionales en lo correspondiente a asuntos internos. Uso de los dispositivos de enlace para fomentar la adaptación mutua y una descentralización selectiva.

\item[Burocracia divisionalizada] Se trata de una serie de entidades semiautónomas acopladas mediante una estructura administrativa central. Estas entidades se denominan divisiones. La administración que las reúne se denomina sede central. El ámbito de control del ápice estratégico puede ser bastante amplio. La sede central permite a las divisiones autonomía casi completa para tomar sus propias divisiones, controlando posteriormente los resultados. Mecanismos de coordinación:Normalización de \textit{output}. Las divisiones adoptan estructuras de burocracia maquinal.
\end{description}




\end{document}